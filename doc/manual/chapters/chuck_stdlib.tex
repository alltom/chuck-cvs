\chapter{Standard Libraries API}

ChucK Standard Libraries API 
these libraries are provide by default with ChucK - new ones can also be 
imported with ChucK dynamic linking (soon to be documented...). 
The existing libraries are organized by namespaces in ChucK. 
They are as follows. 


\chucknamespace{std}

{\bf std} is a standard library in ChucK, which includes utility functions, random number generation, unit conversions and absolute value.

\example

\begin{verbatim}
    /* a simple example... */
    // infinite time-loop
    while( true )
    {
        // generate random float (and print)
        <<< std.rand2f( 100.0, 1000.0 ) >>>;
        // wait a bit
        50::ms => now;
    }
\end{verbatim}


\chuckmultifunction{ int abs ( int value );}{
returns absolute value of integer
}


\chuckmultifunction{ float fabs ( float value );}{
returns absolute value of floating point number
}

\chuckmultifunction{ int rand ( );}{
generates random integer
}


\chuckmultifunction{ int rand2 ( int min, int max );}{
generates random integer in the range [min, max]
}


\chuckmultifunction{ float randf ( );}{
generates random floating point number in the range [-1, 1]
}


\chuckmultifunction{ float rand2f ( float min, float max );}{
generates random floating point number in the range [min, max]
}

\chuckmultifunction{float srand (int value)}{
seeds value for random number generation
}
\chuckmultifunction{ float sgn ( float value );}{
computes the sign of the input as -1.0 (negative), 0 (zero), or 1.0 (positive)
}


\chuckmultifunction{ int system ( string cmd );}{
pass a command to be executed in the shell
}


\chuckmultifunction{ int atoi ( string value );}{
converts ascii (string) to integer (int)
}


\chuckmultifunction{ float atof ( string value );}{
converts ascii (string) to floating point value (float)
}

\chuckmultifunction{ string getenv ( string value );}{
returns the value of an environment variable, such as of "PATH"
}


\chuckmultifunction{ int setenv ( string key, string value );}{
sets environment variable named 'key' to 'value'
}


\chuckmultifunction{ float mtof ( float value );}{
converts a MIDI note number to frequency (Hz)\\
note the input value is of type 'float' (supports fractional note number)
}


\chuckmultifunction{ float ftom ( float value );}{
converts frequency (Hz) to MIDI note number space
}


\chuckmultifunction{ float powtodb ( float value );}{
converts signal power ratio to decibels (dB)
}


\chuckmultifunction{ float rmstodb ( float value );}{
converts linear amplitude to decibels (dB)
}

\chuckmultifunction{ float dbtopow ( float value );}{
converts decibels (dB) to signal power ratio
}


\chuckmultifunction{ float dbtorms ( float value );}{
converts decibles (dB) to linear amplitude
}


\chucknamespace{ machine }

{\bf machine} is ChucK runtime interface to the virtual machine. this interface can be used to manage shreds. They are similar to the On-the-fly Programming Commands, except these are invoked from within a ChucK program, and are subject to the timing mechanism.

\chuckmultifunction{ int add ( string path );}{
compile and spork a new shred from file at 'path' into the VM now\\
returns the shred ID\\
(see example/machine.ck)
}

\chuckmultifunction{ int spork ( string path );}{
same as add/+
}

\chuckmultifunction{ int remove ( int id );}{

remove shred from VM by shred ID (returned by add/spork)
}

\chuckmultifunction{ int replace ( int id, string path );}{
replace shred with new shred from file\\
returns shred ID , or 0 on error
}

\chuckmultifunction{ int status ( );}{
display current status of VM \\
same functionality as status/\^ \\
(see example/status.ck)
}

\chuckmultifunction{ void crash ( );}{
iterally causes the VM to crash. the very last resort; use with care. Thanks.
}


\chucknamespace{ math }


{\bf math} contains the standard math functions. all trignometric functions expects angles to be in radians. 

\example

\begin{verbatim}
    // print sine of pi/2
    <<<< math.sin( pi / 2.0 ) >>>;
\end{verbatim}

\chuckmultifunction{ float sin ( float x );}{
computes the sine of x
}


\chuckmultifunction{ float cos ( float x );}{
computes the cosine of x
}


\chuckmultifunction{ float tan ( float x );}{
computes the tangent of x
}

\chuckmultifunction{ float asin ( float x );}{
computes the arc sine of x
}


\chuckmultifunction{ float acos ( float x );}{
computes the arc cosine of x
}


\chuckmultifunction{ float atan ( float x );}{
computes the arc tangent of x
}


\chuckmultifunction{ float atan2 ( float x );}{
computes the principal value of the arc tangent of y/x, using the signs of both arguments to determine the quadrant of the return value
}


\chuckmultifunction{ float sinh ( float x );}{
computes the hyperbolic sine of x
}


\chuckmultifunction{ float cosh ( float x );}{
computes the hyperbolic cosine of x
}


\chuckmultifunction{ float tanh ( float x );}{
computes the hyperbolic tangent of x
}


\chuckmultifunction{ float hypot ( float x, float y );}{
computes the euclidean distance of the orthogonal vectors (x,0) and (0,y)
}


\chuckmultifunction{ float pow ( float x, float y );}{
computes x taken to the y-th power
}


\chuckmultifunction{ float sqrt ( float x );}{
computes the nonnegative square root of x (x must be >= 0)
}


\chuckmultifunction{ float exp ( float x );}{
computes e\^x, the base-e exponential of x
}


\chuckmultifunction{ float log ( float x );}{
computes the natural logarithm of x
}


\chuckmultifunction{ float log2 ( float x );}{
computes the logarithm of x to base 2
}


\chuckmultifunction{ float log10 ( float x );}{
computes the logarithm of x to base 10
}

\chuckmultifunction{ float floor ( float x );}{
round to largest integral value (returned as float) not greater than x
}


\chuckmultifunction{ float ceil ( float x );}{
round to smallest integral value (returned as float) not less than x
}


\chuckmultifunction{ float round ( float x );}{
round to nearest integral value (returned as float)
}


\chuckmultifunction{ float trunc ( float x );}{
round to largest integral value (returned as float) no greater in magnitude than x
}


\chuckmultifunction{ float fmod ( float x, float y );}{
computes the floating point remainder of x / y
}


\chuckmultifunction{ float remainder ( float x, float y );}{
computes the value r such that r = x - n * y, where n is the integer nearest the exact value of x / y. If there are two integers closest to x / y, n shall be the even one. If r is zero, it is given the same sign as x
}


\chuckmultifunction{ float min ( float x, float y );}{
choose lesser of two values
}

\chuckmultifunction{ float max ( float x, float y );}{
choose greater of two values
}


\chuckmultifunction{ int nextpow2 ( int n );}{
computes the integeral (returned as int) smallest power of 2 greater than the value of x
}


\chuckmultifunction{ float pi ( );}{
(currently disabled - use pi)
}


\chuckmultifunction{ float twopi ( );}{
(currently disabled)
}


\chuckmultifunction{ float e ( );}{
(currently disabled - use math.exp(1) )
}

\chuckmultifunction{int isinf (float);}{
infinity test
}

\chuckmultifunction{int isnan (float)}{
NaN test
}
