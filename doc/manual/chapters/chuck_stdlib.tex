\chapter{Standard Libraries API}

ChucK Standard Libraries API 
these libraries are provide by default with ChucK - new ones can also be 
imported with ChucK dynamic linking (soon to be documented...). 
The existing libraries are organized by namespaces in ChucK. 
They are as follows. 


\chucknamespace{std}

the standard library in ChucK. \\
\example

\begin{verbatim}
        std.rand2f( 100.0, 1000.0 ) => stdout;
\end{verbatim}

\chuckfunction{ float abs ( float value );}


\chuckfunction{ int rand ( );}


\chuckfunction{ int rand2 ( int min, int max );}


\chuckfunction{ float randf ( );}


\chuckfunction{ float rand2f ( float min, float max );}


\chuckfunction{ float sgn ( float value );}


\chuckfunction{ int system ( string cmd );}


\chuckfunction{ int atoi ( string value );}


\chuckfunction{ float atof ( string value );}


\chuckfunction{ string getenv ( string value );}


\chuckfunction{ int setenv ( string key, string value );}


\chuckfunction{ float mtof ( float value );}


\chuckfunction{ float ftom ( float value );}


\chuckfunction{ float powtodb ( float value );}


\chuckfunction{ float rmstodb ( float value );}


\chuckfunction{ float dbtopow ( float value );}


\chuckfunction{ float dbtorms ( float value );}


\chucknamespace{ machine }

ChucK runtime interface to the virtual machine. this interface can be used to manage shreds. 
They are similar to the On-the-fly Programming Commands, except these are invoked from 
within a ChucK program, and are accessible to the timing mechanism. 

\chuckmultifunction{ int add ( string path );}{
compile and spork a new shred from file at 'path' into the VM now\\
returns the shred ID\\
(see example/machine.ck)
}

\chuckmultifunction{ int spork ( string path );}{
same as add
}

\chuckmultifunction{ int remove ( int id );}{

remove shred from VM by shred ID (returned by add/spork)
}

\chuckmultifunction{ int replace ( int id, string path );}{
replace shred with new shred from file\\
returns shred ID , or 0 on error
}

\chuckmultifunction{ int status ( );}{
display current status of VM\\
(see example/status.ck)
}

\chucknamespace{ math }

standard math functions 


\example

\begin{verbatim}
        math.sin( math.pi /2.0 ) => stdout;
\end{verbatim}

\chuckfunction{ float sin ( float x );}


\chuckfunction{ float cos ( float x );}


\chuckfunction{ float tan ( float x );}


\chuckfunction{ float asin ( float x );}


\chuckfunction{ float acos ( float x );}


\chuckfunction{ float atan ( float x );}


\chuckfunction{ float atan2 ( float x );}


\chuckfunction{ float sinh ( float x );}


\chuckfunction{ float cosh ( float x );}


\chuckfunction{ float tanh ( float x );}


\chuckfunction{ float hypot ( float x, float y );}


\chuckfunction{ float pow ( float x, float y );}


\chuckfunction{ float sqrt ( float x );}


\chuckfunction{ float exp ( float x );}


\chuckfunction{ float log ( float x );}


\chuckfunction{ float log2 ( float x );}


\chuckfunction{ float log10 ( float x );}


\chuckfunction{ float floor ( float x );}


\chuckfunction{ float ceil ( float x );}


\chuckfunction{ float round ( float x );}


\chuckfunction{ float trunc ( float x );}


\chuckfunction{ float fmod ( float x, float y );}


\chuckfunction{ float remainder ( float x, float y );}


\chuckfunction{ float min ( float x, float y );}


\chuckfunction{ float max ( float x, float y );}


\chuckfunction{ int nextpow2 ( int n );}


\chuckfunction{ float pi ( );}


\chuckfunction{ float twopi ( );}


\chuckfunction{ float e ( );}