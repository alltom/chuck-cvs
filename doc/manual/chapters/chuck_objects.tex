\chapter{Objects}

ChucK's object class system is very similar to Java. Objects can contain member data and member functions. ChucK allows you to create a class and to extend existing classes. 

To declare a class you use the reserved work {\it class} followed by the name of the class you are going to create. 


\begin{verbatim}
class MyClass
{
    // member data
    int myvalue;


    // member function
    fun int get()
    {
        return m_value;
    }
}

\end{verbatim}

Instantiating a class is done similar to Java. First we give the name of the class and then the name of the instance you wish to create. 

\begin{verbatim}
MyClass myinstance;
\end{verbatim}

We can now access the methods of {\it myinstance} the same way we access the methods of ugens. 


\begin{verbatim}
myinstance.get() => int somevariable;
\end{verbatim}



\begin{verbatim}
class MyChild extend MyClass
{
    fun int get( int mul )
    {
        return m_value * mul;
    }
}

\end{verbatim}


Doing useful things with objects