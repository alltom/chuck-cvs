\section{Conventions}

ChucK compiles under many different operating systems. Each of these has their own ``features'' that make the experience of working with ChucK slightly different. This chapter will outline some of these differences. 

ChucK is a terminal application (no, ChucK is not sick, but it may cause nausea, stress, frustration...). You will need to know how to access and navigate in the terminal.

\subsection*{Finding the Terminal}

OS X

The terminal is located in the Utilities/ folder in the Applications/ folder of your hard drive. Double click on Terminal. You can click and hold on the icon in the Dock and select the ``Keep in Dock'' option. Now the Terminal application will be conveniently located in the Dock. 

Windows


Linux

If you are reading this then you are required to return your tinfoil hat.


\subsection*{Navigating in the Terminal}

You will need to know how to get around in the terminal and run commands. 


OS X

The first command to use is {\bf pwd}, which is short for ``Print Working Directory''. 

\chuckterm{\prompt pwd}

LS 

MAN

CD


Windows

