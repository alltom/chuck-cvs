\chapter{Installation}

We tried to make ChucK as easy as possible to build and re-use. All 
sources files - headers source for compiler, vm, and audio engine - are 
in the same directory (they run in the same address space anyway). Platforms 
differences are abstracted to the lowest level (in part thanks to Gary 
Scavone). None of the compiler/vm has any OS-depedent code.

There are also pre-compiled executables available for OS X and Windows.

\section{Binary Installation}

The binary distributions include a directory called bin/ that contains the precompiled binary of ChucK for your operating system. The binary distribution is a great introduction to ChucK but once you are comfortable we suggest that you compile ChucK on your own system so that it is optimized for your system. 

\subsection{OS X}
1. The terminal is located in the Utilities/ folder in the Applications/ folder of your hard drive. In the terminal go to the bin/ directory (replace chuck-x.x.x.x-exe with the actual directory name):

\chuckterm{
   \prompt cd chuck-x.x.x.x-exe/bin
}

2. Install it using the script included.

\chuckterm{
    \prompt sudo cp chuck /usr/bin/
}

(enter password when prompted)

\chuckterm{    
    \prompt sudo chmod 755 /usr/bin/chuck
}

Now you should be able to run 'chuck' from any directory.

3. Test to make sure it is was installed properly.

\chuckterm{
    \prompt chuck
}

You should see the following message (which is the correct behavior):

\chuckterm{
    [chuck]: no input files... (try --help)
}


\subsection{Windows}

1. Place chuck.exe (found in the 'bin' folder) into c:\textbackslash windowss\textbackslash system32\textbackslash

2. Open a command window found in start-\textgreater run

\includegraphics{images/startmenu}

3. Type cmd and press return

\includegraphics{images/cmd}

4. Type chuck and press return, you should see:

\chuckterm{
    \prompt chuck
    
    [chuck]: no input files... (try --help)
}

\section{Source Installation}

To build chuck from the source (from scratch): 

1. Go to the src/ directory (replace chuck-x.x.x.x with the actual 
directory name):

\chuckterm{
   \prompt cd chuck-x.x.x.x/src/
}


2. If you type 'make' here, you should get the following message:

\chuckterm{
   \prompt make\\
   \chuckbuild: please use one of the following configurations:\\
       make osx, make osx-intel, make linux-oss, make linux-alsa, make linux-jack, make win32
}

Now, type the command corresponding to your platform... 

for example, for MacOS X:

\chuckterm{\prompt make osx}

for example, for Windows:

\chuckterm{\prompt make win32}

3. If you would like to install chuck (cp into /usr/bin by default). If you 
don't like the destination, simply edit the makefile under `install', or 
skip this step altogether. (we recommend putting it somewhere in your 
path, it makes on-the-fly programming easier)

\chuckterm{
   \# (optional: edit the makefile first)\\
   \prompt make install
}

You may need to have administrator privileges in order to install ChucK. If you 
have admin access then you can use the sudo command to install.

\chuckterm{
   \prompt sudo make install
}

4. If you haven't gotten any egregious error messages up to this point, 
then you should be done! There should be a `chuck' executable in the 
current directory. For a quick sanity check, execute the following (use 
`./chuck' if chuck is not in your path), and see if you get the same output:

\chuckterm{
   \prompt chuck
   [chuck]: no input files...
}

(if you do get error messages during compilation, or you run into some 
other problem - please let us know and we will do our best to provide 
support) 

\rule{1in}{.5pt}

You are ready to ChucK. If this is your first time programming in 
ChucK, you may want to look at the documentation, or take the ChucK 
Tutorial ( \href{http://chuck.cs.princeton.edu/doc}{http://chuck.cs.princeton.edu/doc/}). 

ThanK you very much. Go forth and ChucK - email us for support or to make a 
suggestion or to call us idiots.

Ge + Perry
 
