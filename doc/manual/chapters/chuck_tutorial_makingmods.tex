\section{Modifying Basic Patches}

We have a basic patch running in ChucK but it still doesn't sound very good. In this chapter we will cover some simple ways to rectify that problem. ChucK allows one to quickly make modifications to patches that can drastically change the sound.

First what we can do is change the type of our oscillator. There are many different oscillators available to use: sinosc (sine wave), sawosc (sawtooth), sqrosc (square wave) and  pulseosc (pulse wave). We can simply change the type of oscillator just like below. 

\begin{verbatim}
        sawosc s => dac;
\end{verbatim}

Try changing the oscillator to all of the different types and a get a feel for how they sound. When changing the different Ugens always be sure to check the rest of your patches so that the parameter calls are valid. If you were to use the {\bf .width} method of pulseosc and others on a sinosc ChucK will complain. You can comment out lines that are temporarily broken by using double slashes (//).

Now let's add some effects to our patch. ChucK has many different standard effects that can be added to Ugen chains. The simplest effect we can add is an amplifier. In ChucK, this object is {\bf gain}.

\begin{verbatim}
        sawosc s => gain g => dac;
\end{verbatim}

Now we can change the parameters of our effect. {\bf gain} has a parameter {\bf .gain} that can be used to change the gain  of signals passing through the object. Let's go about changing the gain.

\begin{verbatim}
	.5 => g.gain;
\end{verbatim}

This is redundant. All Ugens have the ability to change their gain in a similar manner. 

\begin{verbatim}
	.5 => s.gain;
\end{verbatim}

However, this is useful when we have multiple Ugens connect to one place. If we were to connect 2 oscillators to the {\bf dac} then we will get distortion. By default, these oscillators oscillate between -1 and 1. When they are connected to the same input they are added, so now they go between -2 and 2. This will clip our output. What to do? {\bf gain} to the rescue!

\begin{verbatim}
	sinosc s1 => gain g => dac;
	sinsocs s2 => g;
	.5 => g.gain;
\end{verbatim}

Now our oscillators are scaled between -1 and 1 and everything is right in the world.

\subsection{Having More Fun}

More effects were promised, now you will see some in action. Again, one of the wonders of ChucK is how easy it is to change Ugens. 