\chapter{On-the-fly Programming Commands}

These are used for on-the-fly programming (see http://on-the-fly.cs.princeton.edu). By default, this requires that a ChucK virtual machine be already running on the localhost. It communicates via sockets to add/remove/replace shreds in the VM, and to query VM state. The simplest way to set up a ChucK virtual machine to accept these commands is by starting an empty VM with \doubledash loop: 

 \chuckterm{\prompt  chuck \doubledash loop}

this will start a VM, looping (and advancing time), waiting for incoming commands. Successive invocations of `chuck' with the appropriate commands will communicate with this listener VM. 

(for remote operations over TCP, see below) 


{\bf \doubledash add / +}\\
adds new shreds from source files to the listener VM. this process then exits.  for example: 

\chuckterm{\prompt  chuck + foo.ck bar.ck}

integrates foo.ck and bar.ck into the listener VM. the shreds are internally responsible for finding about the timing and other shreds via the timing mechanism and vm interface. 

{\bf \doubledash remove / - }\\
removes existing shreds from the VM by ID. how to find out about the id? (see \doubledash status below) for example: 

 \chuckterm{\prompt  chuck - 2 3 8}

removes shred 2, 3, 8. 

{\bf \doubledash replace / =}\\
replace existing shred with a new shred. for example: 

\chuckterm{\prompt  chuck = 2 foo.ck}

replaces shred 2 with foo.ck 

{\bf \doubledash status / \^{} }\\
queries the status of the VM - output on the listener VM. for example: 

\chuckterm{\prompt  chuck \^{}}

this prints the internal shred start at the listener VM, something like: 

\chuckterm{
 [chuck](VM): status (now == 0h:2m:34s) ...\\
   \hspace*{0.4in}[shred id]: 1  [source]: foo.ck  [sporked]: 21.43s ago \\
   \hspace*{0.4in}[shred id]: 2  [source]: bar.ck  [sporked]: 28.37s ago 
}


{\bf \doubledash time}\\
prints out the value of now on the listener VM. for example:

 \chuckterm{\prompt  chuck \doubledash time}

something like: 

\chuckterm{
 [chuck](VM): the value of now:
\hspace*{0.4in} now = 403457 (samp)\\
\hspace*{0.6in}     = 9.148685 (second)\\
\hspace*{0.6in}     = 0.152478 (minute)\\
\hspace*{0.6in}     = 0.002541 (hour)\\
\hspace*{0.6in}        = 0.000106 (day)\\
\hspace*{0.6in}        = 0.000015 (week)
}


{\bf \doubledash kill}\\
semi-gracefully kills the listener VM - removes all shreds first. 


{\bf \doubledash remote / \@}\\
specifies where to send the on-the-fly command.  must appear in the command line before any on-the-fly commands.  for example: 


\chuckterm{
 \prompt  chuck @192.168.1.1 + foo.ck bar.ck\\
    \hspace*{0.4in}(or)\\
 \prompt  chuck @foo.bar.com -p8888 + foo.ck bar.ck}

sends foo.ck and bar.ck to VM at 192.168.1.1 or foo.bar.com:8888
