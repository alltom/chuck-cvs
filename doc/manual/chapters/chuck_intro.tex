\chapter{Intro-ChucK-tion} 

{\bf what is it:} ChucK is a general-purpose programming language, intended for 
real-time audio synthesis and graphics/multimedia programming.  It 
introduces a truly concurrent programming model that embeds timing 
directly in the program flow.  Other potentially useful features include 
the ability to write/change programs on-the-fly.

{\bf who it is for:} audio/multimedia researchers, developers, composers, and performers

\underline{supported platforms:}
\begin{itemize}
 \item MacOS X (CoreAudio)
 \item Linux (ALSA/OSS/Jack)
 \item Windows/Cygwin (DirectSound)
 \item SGI (coming soon)
\end{itemize}

\section*{On-the-fly Programming}
On-the-fly programming is a style of programming in which the programmer/performer/composer 
augments and modifies the program while it is running, without stopping or restarting, 
in order to assert expressive, programmable control for performance, composition, and 
experimentation at run-time. Because of the fundamental powers of programming languages, 
we believe the technical and aesthetic aspects of on-the-fly programming are worth exploring.